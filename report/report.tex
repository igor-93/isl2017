%% bare_conf.tex\
%% V1.3\
%% 2007/01/11\
%% by Michael Shell\
%% See:\
%% http://www.michaelshell.org/\
%% for current contact information.\
%%\
%% This is a skeleton file demonstrating the use of IEEEtran.cls\
%% (requires IEEEtran.cls version 1.7 or later) with an IEEE conference paper.\
%%\
%% Support sites:\
%% http://www.michaelshell.org/tex/ieeetran/\
%% http://www.ctan.org/tex-archive/macros/latex/contrib/IEEEtran/\
%% and\
%% http://www.ieee.org/\
\
%%*************************************************************************\
%% Legal Notice:\
%% This code is offered as-is without any warranty either expressed or\
%% implied; without even the implied warranty of MERCHANTABILITY or\
%% FITNESS FOR A PARTICULAR PURPOSE! \
%% User assumes all risk.\
%% In no event shall IEEE or any contributor to this code be liable for\
%% any damages or losses, including, but not limited to, incidental,\
%% consequential, or any other damages, resulting from the use or misuse\
%% of any information contained here.\
%%\
%% All comments are the opinions of their respective authors and are not\
%% necessarily endorsed by the IEEE.\
%%\
%% This work is distributed under the LaTeX Project Public License (LPPL)\
%% ( http://www.latex-project.org/ ) version 1.3, and may be freely used,\
%% distributed and modified. A copy of the LPPL, version 1.3, is included\
%% in the base LaTeX documentation of all distributions of LaTeX released\
%% 2003/12/01 or later.\
%% Retain all contribution notices and credits.\
%% ** Modified files should be clearly indicated as such, including  **\
%% ** renaming them and changing author support contact information. **\
%%\
%% File list of work: IEEEtran.cls, IEEEtran_HOWTO.pdf, bare_adv.tex,\
%%                    bare_conf.tex, bare_jrnl.tex, bare_jrnl_compsoc.tex\
%%*************************************************************************\
\
% *** Authors should verify (and, if needed, correct) their LaTeX system  ***\
% *** with the testflow diagnostic prior to trusting their LaTeX platform ***\
% *** with production work. IEEE's font choices can trigger bugs that do  ***\
% *** not appear when using other class files.                            ***\
% The testflow support page is at:\
% http://www.michaelshell.org/tex/testflow/\
\
\
\
% Note that the a4paper option is mainly intended so that authors in\
% countries using A4 can easily print to A4 and see how their papers will\
% look in print - the typesetting of the document will not typically be\
% affected with changes in paper size (but the bottom and side margins will).\
% Use the testflow package mentioned above to verify correct handling of\
% both paper sizes by the user's LaTeX system.\
%\
% Also note that the "draftcls" or "draftclsnofoot", not "draft", option\
% should be used if it is desired that the figures are to be displayed in\
% draft mode.\
%\
\\documentclass[conference]\{IEEEtran\}\
\\usepackage\{blindtext, graphicx\}\
% Add the compsoc option for Computer Society conferences.\
%\
% If IEEEtran.cls has not been installed into the LaTeX system files,\
% manually specify the path to it like:\
% \\documentclass[conference]\{../sty/IEEEtran\}\
\
\
\
\
\
% Some very useful LaTeX packages include:\
% (uncomment the ones you want to load)\
\
\
% *** MISC UTILITY PACKAGES ***\
%\
%\\usepackage\{ifpdf\}\
% Heiko Oberdiek's ifpdf.sty is very useful if you need conditional\
% compilation based on whether the output is pdf or dvi.\
% usage:\
% \\ifpdf\
%   % pdf code\
% \\else\
%   % dvi code\
% \\fi\
% The latest version of ifpdf.sty can be obtained from:\
% http://www.ctan.org/tex-archive/macros/latex/contrib/oberdiek/\
% Also, note that IEEEtran.cls V1.7 and later provides a builtin\
% \\ifCLASSINFOpdf conditional that works the same way.\
% When switching from latex to pdflatex and vice-versa, the compiler may\
% have to be run twice to clear warning/error messages.\
\
\
\
\
\
\
% *** CITATION PACKAGES ***\
%\
%\\usepackage\{cite\}\
% cite.sty was written by Donald Arseneau\
% V1.6 and later of IEEEtran pre-defines the format of the cite.sty package\
% \\cite\{\} output to follow that of IEEE. Loading the cite package will\
% result in citation numbers being automatically sorted and properly\
% "compressed/ranged". e.g., [1], [9], [2], [7], [5], [6] without using\
% cite.sty will become [1], [2], [5]--[7], [9] using cite.sty. cite.sty's\
% \\cite will automatically add leading space, if needed. Use cite.sty's\
% noadjust option (cite.sty V3.8 and later) if you want to turn this off.\
% cite.sty is already installed on most LaTeX systems. Be sure and use\
% version 4.0 (2003-05-27) and later if using hyperref.sty. cite.sty does\
% not currently provide for hyperlinked citations.\
% The latest version can be obtained at:\
% http://www.ctan.org/tex-archive/macros/latex/contrib/cite/\
% The documentation is contained in the cite.sty file itself.\
\
\
\
\
\
\
% *** GRAPHICS RELATED PACKAGES ***\
%\
\\ifCLASSINFOpdf\
  % \\usepackage[pdftex]\{graphicx\}\
  % declare the path(s) where your graphic files are\
  % \\graphicspath\{\{../pdf/\}\{../jpeg/\}\}\
  % and their extensions so you won't have to specify these with\
  % every instance of \\includegraphics\
  % \\DeclareGraphicsExtensions\{.pdf,.jpeg,.png\}\
\\else\
  % or other class option (dvipsone, dvipdf, if not using dvips). graphicx\
  % will default to the driver specified in the system graphics.cfg if no\
  % driver is specified.\
  % \\usepackage[dvips]\{graphicx\}\
  % declare the path(s) where your graphic files are\
  % \\graphicspath\{\{../eps/\}\}\
  % and their extensions so you won't have to specify these with\
  % every instance of \\includegraphics\
  % \\DeclareGraphicsExtensions\{.eps\}\
\\fi\
% graphicx was written by David Carlisle and Sebastian Rahtz. It is\
% required if you want graphics, photos, etc. graphicx.sty is already\
% installed on most LaTeX systems. The latest version and documentation can\
% be obtained at: \
% http://www.ctan.org/tex-archive/macros/latex/required/graphics/\
% Another good source of documentation is "Using Imported Graphics in\
% LaTeX2e" by Keith Reckdahl which can be found as epslatex.ps or\
% epslatex.pdf at: http://www.ctan.org/tex-archive/info/\
%\
% latex, and pdflatex in dvi mode, support graphics in encapsulated\
% postscript (.eps) format. pdflatex in pdf mode supports graphics\
% in .pdf, .jpeg, .png and .mps (metapost) formats. Users should ensure\
% that all non-photo figures use a vector format (.eps, .pdf, .mps) and\
% not a bitmapped formats (.jpeg, .png). IEEE frowns on bitmapped formats\
% which can result in "jaggedy"/blurry rendering of lines and letters as\
% well as large increases in file sizes.\
%\
% You can find documentation about the pdfTeX application at:\
% http://www.tug.org/applications/pdftex\
\
\
\
\
\
% *** MATH PACKAGES ***\
%\
%\\usepackage[cmex10]\{amsmath\}\
% A popular package from the American Mathematical Society that provides\
% many useful and powerful commands for dealing with mathematics. If using\
% it, be sure to load this package with the cmex10 option to ensure that\
% only type 1 fonts will utilized at all point sizes. Without this option,\
% it is possible that some math symbols, particularly those within\
% footnotes, will be rendered in bitmap form which will result in a\
% document that can not be IEEE Xplore compliant!\
%\
% Also, note that the amsmath package sets \\interdisplaylinepenalty to 10000\
% thus preventing page breaks from occurring within multiline equations. Use:\
%\\interdisplaylinepenalty=2500\
% after loading amsmath to restore such page breaks as IEEEtran.cls normally\
% does. amsmath.sty is already installed on most LaTeX systems. The latest\
% version and documentation can be obtained at:\
% http://www.ctan.org/tex-archive/macros/latex/required/amslatex/math/\
\
\
\
\
\
% *** SPECIALIZED LIST PACKAGES ***\
%\
%\\usepackage\{algorithmic\}\
% algorithmic.sty was written by Peter Williams and Rogerio Brito.\
% This package provides an algorithmic environment fo describing algorithms.\
% You can use the algorithmic environment in-text or within a figure\
% environment to provide for a floating algorithm. Do NOT use the algorithm\
% floating environment provided by algorithm.sty (by the same authors) or\
% algorithm2e.sty (by Christophe Fiorio) as IEEE does not use dedicated\
% algorithm float types and packages that provide these will not provide\
% correct IEEE style captions. The latest version and documentation of\
% algorithmic.sty can be obtained at:\
% http://www.ctan.org/tex-archive/macros/latex/contrib/algorithms/\
% There is also a support site at:\
% http://algorithms.berlios.de/index.html\
% Also of interest may be the (relatively newer and more customizable)\
% algorithmicx.sty package by Szasz Janos:\
% http://www.ctan.org/tex-archive/macros/latex/contrib/algorithmicx/\
\
\
\
\
% *** ALIGNMENT PACKAGES ***\
%\
%\\usepackage\{array\}\
% Frank Mittelbach's and David Carlisle's array.sty patches and improves\
% the standard LaTeX2e array and tabular environments to provide better\
% appearance and additional user controls. As the default LaTeX2e table\
% generation code is lacking to the point of almost being broken with\
% respect to the quality of the end results, all users are strongly\
% advised to use an enhanced (at the very least that provided by array.sty)\
% set of table tools. array.sty is already installed on most systems. The\
% latest version and documentation can be obtained at:\
% http://www.ctan.org/tex-archive/macros/latex/required/tools/\
\
\
%\\usepackage\{mdwmath\}\
%\\usepackage\{mdwtab\}\
% Also highly recommended is Mark Wooding's extremely powerful MDW tools,\
% especially mdwmath.sty and mdwtab.sty which are used to format equations\
% and tables, respectively. The MDWtools set is already installed on most\
% LaTeX systems. The lastest version and documentation is available at:\
% http://www.ctan.org/tex-archive/macros/latex/contrib/mdwtools/\
\
\
% IEEEtran contains the IEEEeqnarray family of commands that can be used to\
% generate multiline equations as well as matrices, tables, etc., of high\
% quality.\
\
\
%\\usepackage\{eqparbox\}\
% Also of notable interest is Scott Pakin's eqparbox package for creating\
% (automatically sized) equal width boxes - aka "natural width parboxes".\
% Available at:\
% http://www.ctan.org/tex-archive/macros/latex/contrib/eqparbox/\
\
\
\
\
\
% *** SUBFIGURE PACKAGES ***\
%\\usepackage[tight,footnotesize]\{subfigure\}\
% subfigure.sty was written by Steven Douglas Cochran. This package makes it\
% easy to put subfigures in your figures. e.g., "Figure 1a and 1b". For IEEE\
% work, it is a good idea to load it with the tight package option to reduce\
% the amount of white space around the subfigures. subfigure.sty is already\
% installed on most LaTeX systems. The latest version and documentation can\
% be obtained at:\
% http://www.ctan.org/tex-archive/obsolete/macros/latex/contrib/subfigure/\
% subfigure.sty has been superceeded by subfig.sty.\
\
\
\
%\\usepackage[caption=false]\{caption\}\
%\\usepackage[font=footnotesize]\{subfig\}\
% subfig.sty, also written by Steven Douglas Cochran, is the modern\
% replacement for subfigure.sty. However, subfig.sty requires and\
% automatically loads Axel Sommerfeldt's caption.sty which will override\
% IEEEtran.cls handling of captions and this will result in nonIEEE style\
% figure/table captions. To prevent this problem, be sure and preload\
% caption.sty with its "caption=false" package option. This is will preserve\
% IEEEtran.cls handing of captions. Version 1.3 (2005/06/28) and later \
% (recommended due to many improvements over 1.2) of subfig.sty supports\
% the caption=false option directly:\
%\\usepackage[caption=false,font=footnotesize]\{subfig\}\
%\
% The latest version and documentation can be obtained at:\
% http://www.ctan.org/tex-archive/macros/latex/contrib/subfig/\
% The latest version and documentation of caption.sty can be obtained at:\
% http://www.ctan.org/tex-archive/macros/latex/contrib/caption/\
\
\
\
\
% *** FLOAT PACKAGES ***\
%\
%\\usepackage\{fixltx2e\}\
% fixltx2e, the successor to the earlier fix2col.sty, was written by\
% Frank Mittelbach and David Carlisle. This package corrects a few problems\
% in the LaTeX2e kernel, the most notable of which is that in current\
% LaTeX2e releases, the ordering of single and double column floats is not\
% guaranteed to be preserved. Thus, an unpatched LaTeX2e can allow a\
% single column figure to be placed prior to an earlier double column\
% figure. The latest version and documentation can be found at:\
% http://www.ctan.org/tex-archive/macros/latex/base/\
\
\
\
%\\usepackage\{stfloats\}\
% stfloats.sty was written by Sigitas Tolusis. This package gives LaTeX2e\
% the ability to do double column floats at the bottom of the page as well\
% as the top. (e.g., "\\begin\{figure*\}[!b]" is not normally possible in\
% LaTeX2e). It also provides a command:\
%\\fnbelowfloat\
% to enable the placement of footnotes below bottom floats (the standard\
% LaTeX2e kernel puts them above bottom floats). This is an invasive package\
% which rewrites many portions of the LaTeX2e float routines. It may not work\
% with other packages that modify the LaTeX2e float routines. The latest\
% version and documentation can be obtained at:\
% http://www.ctan.org/tex-archive/macros/latex/contrib/sttools/\
% Documentation is contained in the stfloats.sty comments as well as in the\
% presfull.pdf file. Do not use the stfloats baselinefloat ability as IEEE\
% does not allow \\baselineskip to stretch. Authors submitting work to the\
% IEEE should note that IEEE rarely uses double column equations and\
% that authors should try to avoid such use. Do not be tempted to use the\
% cuted.sty or midfloat.sty packages (also by Sigitas Tolusis) as IEEE does\
% not format its papers in such ways.\
\
\
\
\
\
% *** PDF, URL AND HYPERLINK PACKAGES ***\
%\
%\\usepackage\{url\}\
% url.sty was written by Donald Arseneau. It provides better support for\
% handling and breaking URLs. url.sty is already installed on most LaTeX\
% systems. The latest version can be obtained at:\
% http://www.ctan.org/tex-archive/macros/latex/contrib/misc/\
% Read the url.sty source comments for usage information. Basically,\
% \\url\{my_url_here\}.\
\
\
\
\
\
% *** Do not adjust lengths that control margins, column widths, etc. ***\
% *** Do not use packages that alter fonts (such as pslatex).         ***\
% There should be no need to do such things with IEEEtran.cls V1.6 and later.\
% (Unless specifically asked to do so by the journal or conference you plan\
% to submit to, of course. )\
\
\
% correct bad hyphenation here\
\\hyphenation\{op-tical net-works semi-conduc-tor\}\
\
\
\\begin\{document\}\
%\
% paper title\
% can use linebreaks \\\\ within to get better formatting as desired\
\\title\{Dandli, a Web Application Helping Professional Photographers to Increase Their Engagement on Instagram\}\
\
\
% author names and affiliations\
% use a multiple column layout for up to three different\
% affiliations\
\\author\{\\IEEEauthorblockN\{Raphael Inglin\}\
\\IEEEauthorblockA\{Global Information Systems\\\\ETH Z\{\\"u\}rich\\\\\
Email: inglinra@student.ethz.ch\}\
\\and\
\\IEEEauthorblockN\{Igor Pesic\}\
\\IEEEauthorblockA\{Global Information Systems\\\\ ETH\
Z\{\\"u\}rich\\\\\
Email: pesici@student.ethz.ch\}\
\\and\
\\IEEEauthorblockN\{Amirreza Bahreini\}\
\\IEEEauthorblockA\{Global Information Systems\\\\ ETH Z\{\\"u\}rich\\\\\
Email: abahrein@student.ethz.ch\}\}\
\
% conference papers do not typically use \\thanks and this command\
% is locked out in conference mode. If really needed, such as for\
% the acknowledgment of grants, issue a \\IEEEoverridecommandlockouts\
% after \\documentclass\
\
% for over three affiliations, or if they all won't fit within the width\
% of the page, use this alternative format:\
% \
%\\author\{\\IEEEauthorblockN\{Michael Shell\\IEEEauthorrefmark\{1\},\
%Homer Simpson\\IEEEauthorrefmark\{2\},\
%James Kirk\\IEEEauthorrefmark\{3\}, \
%Montgomery Scott\\IEEEauthorrefmark\{3\} and\
%Eldon Tyrell\\IEEEauthorrefmark\{4\}\}\
%\\IEEEauthorblockA\{\\IEEEauthorrefmark\{1\}School of Electrical and Computer Engineering\\\\\
%Georgia Institute of Technology,\
%Atlanta, Georgia 30332--0250\\\\ Email: see http://www.michaelshell.org/contact.html\}\
%\\IEEEauthorblockA\{\\IEEEauthorrefmark\{2\}Twentieth Century Fox, Springfield, USA\\\\\
%Email: homer@thesimpsons.com\}\
%\\IEEEauthorblockA\{\\IEEEauthorrefmark\{3\}Starfleet Academy, San Francisco, California 96678-2391\\\\\
%Telephone: (800) 555--1212, Fax: (888) 555--1212\}\
%\\IEEEauthorblockA\{\\IEEEauthorrefmark\{4\}Tyrell Inc., 123 Replicant Street, Los Angeles, California 90210--4321\}\}\
\
\
\
\
% use for special paper notices\
%\\IEEEspecialpapernotice\{(Invited Paper)\}\
\
\
\
\
% make the title area\
\\maketitle\
\
\
\\begin\{abstract\}\
%\\boldmath\
Many recent studies have shown that the engagement on social media does not depend only on the content but also on the time of posts, hashtags used and overall reaction of the user to the negative or positive comments that they receive. In this paper we present a web application targeted primarily to professional photographers that tackles these topics in a particular manner that has not been done yet for Instagram. For the comment engagement, a state-of-the-art sentiment analysis is carried out on user comments so that comments can be filtered out by their sentiments. A preliminary user study has been carried out to evaluate the user friendliness of the application and the effectiveness of the methods used.\
\\end\{abstract\}\
% IEEEtran.cls defaults to using nonbold math in the Abstract.\
% This preserves the distinction between vectors and scalars. However,\
% if the journal you are submitting to favors bold math in the abstract,\
% then you can use LaTeX's standard command \\boldmath at the very start\
% of the abstract to achieve this. Many IEEE journals frown on math\
% in the abstract anyway.\
\
% Note that keywords are not normally used for peerreview papers.\
\\begin\{IEEEkeywords\}\
Instagram, engagement, photography, best post time, best hashtag, sentiment analysis.\
\\end\{IEEEkeywords\}\
\
\
\
\
\
\
% For peer review papers, you can put extra information on the cover\
% page as needed:\
% \\ifCLASSOPTIONpeerreview\
% \\begin\{center\} \\bfseries EDICS Category: 3-BBND \\end\{center\}\
% \\fi\
%\
% For peerreview papers, this IEEEtran command inserts a page break and\
% creates the second title. It will be ignored for other modes.\
\\IEEEpeerreviewmaketitle\
\
\
\
\\section\{Introduction\}\
\\label\{introduction\}\
\
Unlike many other social media websites, Instagram does not provide any analytic tool for its users. The result is many diverse applications trying to provide information to user to help increase the engagement. The knowledge that the time of posts and hashtags used in a post impact the engagement is not something new or particular to Instagram. However, there is no possibility to clearly state what these are. We believe that this is the case because different Instagram users use Instagram for different reasons and in different context. We try to tackle this problem in a different way as we believe that this information should be customized for a particular user. Instead of suggesting the absolute best time to post (or most effective hashtags) on Instagram, we provide a tool that suggests best time to post (or best hashtags) given a particular targeted audience. \\\\\
Another factor that helps users to increase their engagement is their own engagement toward their comments. Appropriate answers are required from the user to both negative and positive comments they receive as ignoring those comments can change fan's perception. This effect has been shown specially on YouTube. In our web application, we also provide a comment management tool which makes it easier for user to handle these comments. \\\\\
Finally, in addition to the three factors mentioned above, we also provide a tool that helps Instagram users to keep their $\\frac\{\\#Follows\}\{\\#Followers\}$ ratio low. Although we don't know any other application trying to solve this problem, we believe that this is of significant importance for Instagram users. Having a large number of $Follows$ may bring new followers, but at the same time it may indicate that most of your $Followers$ are result of mutual following and that they are not truly interested in you content. So it's important to know who to follow.\\\\\
We primarily focus on professional photographers, but the idea can be easily applied on any other context in the same way. \\\\\
In what follows, we first present the methods used and how they work in section \\ref\{Method\}. Method, then in section \\ref\{Architecture\}. Architecture we present a brief explanation of the technical details of the software and finally in section \\ref\{study\}. User Study we present the result of the preliminary evaluation of the application.\
\
\\subsection\{Related works\}\
To our knowledge, the only application that uses the same methodology for hashtag suggestions and time to post suggestions is TweetAnalyser\\footnote\{\\url\{https://tweetanalyzer.net/\}\} for Tweeter.\\\\\
Regarding the comment management tool, the only tools out there using sentiment analysis on comments are again application for Tweeter and not Instagram.\\\\\
\
\\section\{Method\}\
\\label\{Method\}\
In this section we explain the four main features of \\textfb\{Dandli\} that help to increase the engagement.\
\
\\subsection\{Post/Tag recommendations\} \
\\textfb\{Dandli\} tackles the problem of recommendations by assuming that the content of user's page is relevant to some other famous or role model photographers. This way, by analyzing the one or more of these photographers the app suggests the most effective hashtags or best time slots to post by those users. \\\\\
To get these recommendations, user must add the username of some other professional role model photographers which have very high amount of engagement and the same topic. The latter is very important, as the context in which those professional photographers are posting must be the same as the user. Let's say the user would like to post a picture of a lion in the nature. Then the user must choose some of their role model professional photographers who take photos of only animals and nature and then \\textfb\{Dandli\} gives back the most effective hashtags or times of post of those users both separately and aggregated together. \\\\\
\
To do this analysis, \\textfb\{Dandli\} uses the concept of \\textit\{Post\\_Scores\} for each post of those role model users. The \\textit\{Post\\_Score\} is based on the ratio of number of likes for a post to the number of followers of a user. So the more likes a post gets, and the less followers in general they have, the higher \\textit\{Post\\_Score\}.\
\
\\subsubsection\{Hashtag recommendations\}\
Hashtags are an important element of social media. If chosen right, they can bring the user more audience through the built-in search of Instagram based on hashtags. A very common problem is that a user does not know which hashtags to use. As mentionned earlier, this problem is solved by analyzing the page of the role model users.\\\\\
\
For a specific role model user, an efficient hashtag is computed by a linear combination of total \\textit\{Post\\_Score\}s related to that hashtag (over all posts where it has been used) and the total popularity of the hashtag on Instagram : $$Tag\\_Score = S + \\alpha \\times N$$ where $S$ is the total \\textit\{Post\\_Score\} for the hashtag and $N$ is the total number of times that this hashtag is used onverall on Instagram.\
The first criteria makes sure that the hashtag brings indeed more likes, and the second criteria makes sure that the hashtag is a popular one, and that there is a high chance somebody search it on Instagram and get redirected to user's posts.  The factor $\\alpha$ is chosen heuristically so that the results are relevant (personal enough to the role model user and popular enough that can be used by other users).\
\
\\subsubsection\{Best time to post\}\
Time to Post is also an important element of social media. It can affect the number of likes you get in a significant way as many studies show. \\\\\
We tackle the problem the same way as we tackled the hashtag problem: Best time slots are based on total  \\textit\{Post\\_Score\} for each time slot of two hours.\\\\\
Of course, this does not mean the best time to post per se, maybe those users never post at Monday evening, thus we won't have any information of how good Monday evening is. But the goal is rather to find out the best times of posts that works well for those users.\
\
\\subsection\{Comment management\}\
One nice feature that is missing on Instagram is the ability to put all recently received comments for all posts, old or new, in one single place to be able then to reply them back. The comment section of \\textfb\{Dandli\} not only provides this, but also using a state-of-the-art sentiment analysis method for English sentences, classifies comments in Negative, Positive or Neutral so that the user can filter them as they wish to prioritize comments that are most important to them. Negative comments specifically need very quick care, be it removing them or replying them back. A user friendly background colors of Green, Red and Yellow are how these sentiments corresponding to each comment are shown.\
\
\\subsection\{Unfollow suggestions\}\
As discussed earlier in section \\ref\{introduction\} Introduction, keeping the ratio $\\frac\{\\#Follows\}\{\\#Followers\}$ low is important. We believe that having a very small number of follows, at the same time of having a very large number of followers motivate a random person to follow a user, as it leaves a better "impression" on them: The content or the user is probably worth following. In the contrary when this ratio closes to 1 (or more) it leaves the impression that most followers are probably the same as the ones being followed and that they are following the user out of mutual interest or respect. \\\\\
Of course theoretically the ideal case is when the number of follows is zero, thus the ratio is zero. But this is not practical for two following reasons. First, user may be interested in following some of their favorite Instagram profiles, just because of the content, that's what all Instagram is about: to follow your favorite users and get feeds from their posts. But the second reason is that many users follow some specific users not out of their interest in their content, but because they wish to get more followers, specifically the followers of the people they follow. This happens specially when there is a famous account with a lot of followers which posts exactly the same topic as the user, this way by following them we may have a better chance of having their followers following us. The problem arises when some of these people that a user starts following get \\textfb\{inactive\} after a period and that they don't post anymore or rarely. Regardless of which of the aforementioned reasons, the user may want to unfollow these users. \\textfb\{Dandli\} provides a tool that ranks the follows based on their inactivity so that the user can always remove inactive and uninteresting follows, thus keeping the mentioned ratio low.\
The inactivity is based on the average number of posts in last two weeks compared to the four weeks before that (6 weeks ago).\
\
\\section\{Architecture\}\
\\label\{Architecture\}\
\
In this project we used the typical architecture for the web applications, namely model-view-control. For that purpose we used Django framework\\footnote\{\\url\{www.djangoproject.com/\}\} as the core of our back-end. For data store we used two databases: SQLite and mongoDB. SQLite stores the information about the logged-in user and who he tracks and follows. The mongoDB then stores all the relevant information about the tracked and followed users (e.g. their recent activity and information needed to calculate post and tag scores). In order to enable the single-page-app we used Django Channels library which runs on top of Redis for the message passing. Next to Django and data store servers, we have two more processes running. The first one is CoreNLP\\footnote\{\\url\{www.stanfordnlp.github.io/CoreNLP/\}\} server, the Stanford's  library for the sentiment analysis that we use for Comment Management. The second one is the job that periodically fetches the number of followers of registered users from Instagram. This information was primarily used in the evaluation phase.\\\\\
\\\\\
On the client side we used, among others, React, Redux and Sass. These enabled us to have the client-side with states which was needed for our singe-page-app. \
\
\\section\{User Study\}\
A preliminary evaluation is carried out in order to get feedback about the user friendliness of the application as well as it's effectiveness to solve the problems related to the engagement. We randomly found some professional photographers on Instagram, sent them an email explaining our project and asked them whether they would like to participate in our user study. Due to lack of time we managed to find only 4 people to participate. The participants were all photographers using Instagram as one of the important mediums of sharing their work. The number of their followers varied from ~600 to ~5000. Participants were required to use their Instagram account as usual for one week. During this first week their number of followers were tracked. Then they were asked to use Dandli and use its features (if convenient) for another week. The same information was tracked. In the end, results of both week are compared to find out whether any significant progress has been made. In addition, the users were required to fill a survey prepared by us to answer questions related to the UI and effectiveness of the methods. The results are as below.\
\
\\subsection\{Tracking results\}\
\
\\subsection\{Survey\}\
Overall all participants were satisfied by their experience. Below are important observations from the answers to our questionnaire and user's comments:\
\
\\subsubsection\{Core Features\}\
\
\\begin\{itemize\}\
    \\item All participants would likely recommend the app to a friend\
    \\item \\textfb\{Dandli\} scores 6.25 out of 10 in average\
    \\item \\textfb\{Dandli\} scores 4.25 out of 5 for ease of use.\
    \\item Each of the features was very important to at least 50\\% of users (hashtag suggestions to 66\\%) except the sentiment analysis\
    \\item Sentiment analysis is mentioned as the least useful feature and even a participant asked for its removal\
\\end\{itemize\}\
\
Some important points are worth mentioning: Users complain about one of the core features of the app, mainly that it takes too much effort to select the "role model" users in order to get Tag/Time suggestions. One user states that even after that, the hashtags could be irrelevant to some of their photos that they are planning to post. Another user with much more experience points out the fact that most of the suggestions are the tags that they already knew, but they were expecting for something different. \\\\\
Another complain was about the fact that the app provides only information and has no interaction with Instagram, like the ability to post. \\\\\
\
\\subsubsection\{UI feedbacks\}\
\\begin\{itemize\}\
    \\item One of the participants mentions that an improvement in the layout of the page, information flow and speed would be a nice step. \
    \\item Two other participants indicated that it would be nicer to have more hashtags after a click on the word cloud, to be able to copy paste more hashtags at the same time. Right now the tool allows only to copy paste one hash tag after the other which of course makes it time consuming.\
    \\item The time suggestions table needs to be wider and maybe more intuitive to read as it seems to some users a bit difficult to read. \
    \\item The number of inactive follows were limited and users could not see many of other suggestions. This must be indeed fixed as it's a very important issue. We should also make the user able to remove some of the suggestions in order not to appear anymore in the list as they may not be interested in unfollowing them for many reasons.\
\\end\{itemize\}\
\
\
\\label\{study\}\
% needed in second column of first page if using \\IEEEpubid\
%\\IEEEpubidadjcol\
\
% An example of a floating figure using the graphicx package.\
% Note that \\label must occur AFTER (or within) \\caption.\
% For figures, \\caption should occur after the \\includegraphics.\
% Note that IEEEtran v1.7 and later has special internal code that\
% is designed to preserve the operation of \\label within \\caption\
% even when the captionsoff option is in effect. However, because\
% of issues like this, it may be the safest practice to put all your\
% \\label just after \\caption rather than within \\caption\{\}.\
%\
% Reminder: the "draftcls" or "draftclsnofoot", not "draft", class\
% option should be used if it is desired that the figures are to be\
% displayed while in draft mode.\
%\
%\\begin\{figure\}[!t]\
%\\centering\
%\\includegraphics[width=2.5in]\{myfigure\}\
% where an .eps filename suffix will be assumed under latex, \
% and a .pdf suffix will be assumed for pdflatex; or what has been declared\
% via \\DeclareGraphicsExtensions.\
%\\caption\{Simulation Results\}\
%\\label\{fig_sim\}\
%\\end\{figure\}\
\
% Note that IEEE typically puts floats only at the top, even when this\
% results in a large percentage of a column being occupied by floats.\
\
\
% An example of a double column floating figure using two subfigures.\
% (The subfig.sty package must be loaded for this to work.)\
% The subfigure \\label commands are set within each subfloat command, the\
% \\label for the overall figure must come after \\caption.\
% \\hfil must be used as a separator to get equal spacing.\
% The subfigure.sty package works much the same way, except \\subfigure is\
% used instead of \\subfloat.\
%\
%\\begin\{figure*\}[!t]\
%\\centerline\{\\subfloat[Case I]\\includegraphics[width=2.5in]\{subfigcase1\}%\
%\\label\{fig_first_case\}\}\
%\\hfil\
%\\subfloat[Case II]\{\\includegraphics[width=2.5in]\{subfigcase2\}%\
%\\label\{fig_second_case\}\}\}\
%\\caption\{Simulation results\}\
%\\label\{fig_sim\}\
%\\end\{figure*\}\
%\
% Note that often IEEE papers with subfigures do not employ subfigure\
% captions (using the optional argument to \\subfloat), but instead will\
% reference/describe all of them (a), (b), etc., within the main caption.\
\
\
% An example of a floating table. Note that, for IEEE style tables, the \
% \\caption command should come BEFORE the table. Table text will default to\
% \\footnotesize as IEEE normally uses this smaller font for tables.\
% The \\label must come after \\caption as always.\
%\
%\\begin\{table\}[!t]\
%% increase table row spacing, adjust to taste\
%\\renewcommand\{\\arraystretch\}\{1.3\}\
% if using array.sty, it might be a good idea to tweak the value of\
% \\extrarowheight as needed to properly center the text within the cells\
%\\caption\{An Example of a Table\}\
%\\label\{table_example\}\
%\\centering\
%% Some packages, such as MDW tools, offer better commands for making tables\
%% than the plain LaTeX2e tabular which is used here.\
%\\begin\{tabular\}\{|c||c|\}\
%\\hline\
%One & Two\\\\\
%\\hline\
%Three & Four\\\\\
%\\hline\
%\\end\{tabular\}\
%\\end\{table\}\
\
\
% Note that IEEE does not put floats in the very first column - or typically\
% anywhere on the first page for that matter. Also, in-text middle ("here")\
% positioning is not used. Most IEEE journals use top floats exclusively.\
% Note that, LaTeX2e, unlike IEEE journals, places footnotes above bottom\
% floats. This can be corrected via the \\fnbelowfloat command of the\
% stfloats package.\
\
\
\
\\section\{Discussion \\& Conclusion\}\
We presented a novel approach to tackle the problem of finding appropriate hashtags for a post, appropriate time to post and increasing the engagement through comments. The approach is unique as the context plays an important role and suggestions depend on this context. Despite the overall satisfaction of the participants in the user study, we learned that the fact that this context must be chosen by the users themselves, mainly choosing the "role models", makes the application cumbersome and hard to use. Users also complained about the fact that once a context ("role models") is chosen then the hashtags are irrelevant to a new kind of posts with different context. Of course they could remove all "role models" and replace them by others which have a relevant context, but this is not user friendly. We could solve this by letting the users to group their "role models" to create different contexts for different kinds of posts, but the problem of choosing the "role models" is still unsolved. We may fix this problem by some image analysis on the new image to post in order to find the context with machine learning techniques and then find the relevant "role models". \\\\\
Sentiment analysis proved to be very useless, mainly because of very inaccurate results. Of course the algorithm itself was not the focus of this project and we just used an already trained model. This could be indeed improved significantly with training the model on specifically Instagram comments. \\\\\
Despite some minor discomfort with the user interface mentioned in the survey, the overall rating for user friendliness was very high and no technical issues happened during the entire user study.\
\
\
\
\
% if have a single appendix:\
%\\appendix[Proof of the Zonklar Equations]\
% or\
%\\appendix  % for no appendix heading\
% do not use \\section anymore after \\appendix, only \\section*\
% is possibly needed\
\
% use appendices with more than one appendix\
% then use \\section to start each appendix\
% you must declare a \\section before using any\
% \\subsection or using \\label (\\appendices by itself\
% starts a section numbered zero.)\
%\
\
% use section* for acknowledgement\
\\section*\{Acknowledgment\}\
\
\
The authors would like to thank first Professor Moira Norrie for her support. She made us believe that it's possible to brain storm an idea, build the idea, and make a simple user study in a very short amount of time. We would like to thank our supervisor Amir E. Sarabadani Tafreshi for his support and valuable remarks throughout the project. \\\\\
Special thanks to all the participants in the user study who helped us make this evaluation possible.\
\
\
% Can use something like this to put references on a page\
% by themselves when using endfloat and the captionsoff option.\
\\ifCLASSOPTIONcaptionsoff\
  \\newpage\
\\fi\
\
\
\
% trigger a \\newpage just before the given reference\
% number - used to balance the columns on the last page\
% adjust value as needed - may need to be readjusted if\
% the document is modified later\
%\\IEEEtriggeratref\{8\}\
% The "triggered" command can be changed if desired:\
%\\IEEEtriggercmd\{\\enlargethispage\{-5in\}\}\
\
% references section\
\
% can use a bibliography generated by BibTeX as a .bbl file\
% BibTeX documentation can be easily obtained at:\
% http://www.ctan.org/tex-archive/biblio/bibtex/contrib/doc/\
% The IEEEtran BibTeX style support page is at:\
% http://www.michaelshell.org/tex/ieeetran/bibtex/\
%\\bibliographystyle\{IEEEtran\}\
% argument is your BibTeX string definitions and bibliography database(s)\
%\\bibliography\{IEEEabrv,../bib/paper\}\
%\
% <OR> manually copy in the resultant .bbl file\
% set second argument of \\begin to the number of references\
% (used to reserve space for the reference number labels box)\
%\\begin\{thebibliography\}\{1\}\
\
%\\bibitem\{IEEEhowto:kopka\}\
%H.~Kopka and P.~W. Daly, \\emph\{A %Guide to \\LaTeX\}, 3rd~ed.\\hskip %1em plus\
%  0.5em minus 0.4em\\relax Harlow, %England: Addison-Wesley, 1999.\
\
%\\end\{thebibliography\}\
\
% biography section\
% \
% If you have an EPS/PDF photo (graphicx package needed) extra braces are\
% needed around the contents of the optional argument to biography to prevent\
% the LaTeX parser from getting confused when it sees the complicated\
% \\includegraphics command within an optional argument. (You could create\
% your own custom macro containing the \\includegraphics command to make things\
% simpler here.)\
%\\begin\{biography\}[\{\\includegraphics[width=1in,height=1.25in,clip,keepaspectratio]\{mshell\}\}]\{Michael Shell\}\
% or if you just want to reserve a space for a photo:\
\
\\begin\{IEEEbiography\}[\{\\includegraphics[width=1in,height=1.25in,clip,keepaspectratio]\{picture\}\}]\{John Doe\}\
\\end\{IEEEbiography\}\
\
% You can push biographies down or up by placing\
% a \\vfill before or after them. The appropriate\
% use of \\vfill depends on what kind of text is\
% on the last page and whether or not the columns\
% are being equalized.\
\
%\\vfill\
\
% Can be used to pull up biographies so that the bottom of the last one\
% is flush with the other column.\
%\\enlargethispage\{-5in\}\
\
\
\
\
% that's all folks\
\\end\{document\}\
\
\
}
